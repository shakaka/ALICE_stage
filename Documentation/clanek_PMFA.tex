\documentclass[10pt,a4paper]{article}

\usepackage[utf8]{inputenc}
% \usepackage[czech]{babel}

\usepackage{caption}
\usepackage{geometry}
\usepackage{graphicx}
\usepackage{subcaption}
\usepackage[obeyDraft]{todonotes}
\usepackage{indentfirst}

\usepackage{hyperref}

\title{Objev antihelia a antitritia  experimentem STAR}
\author{J. Crkovska}

\begin{document}
\maketitle



%%%%%%%%%%%%%%%%%%%%%%%%%%%%%%%%%%%%%%%%%%%%%%%%%%%%%%%%%%%%%%%%%%%%%%%%%%%%%%%%%%%%%%%%%%%%%%%%%%%%%%%%%%

\section{Objevy antičástic}



Pojem antihmoty zavedl roku 1930 britský fyzik Paul Dirac. Všimnul si, že pro vlnovou funkci popisující elektron existují čtyří možná řešení \cite{perkins}. Avšak pouze dvě z nich skutečně odpovídala elektronu. Zbylá dvě byla přiřknuta jeho antičástici -- protějšek z antihmoty o stejné hmotnosti, ale opačném náboji. 
Nicméně až do experimentálního potvrzení antielektronu -- též zvaného pozitron -- v roce 1932 \cite{anderson_1932}  byla antihmota považována pouze za matematický prostředek vysvětlující výše popsaný problém.

Nicméně je zajisté na místě říci, co vlastně antihmota je. Jak již z názvu plyne, jde o jakýsi "protiklad" hmoty. Částice a antičástice mají stejnou hmotnost a dobu života. Naopak nesou opačný elektrický náboj a další additivní kvantová čísla (např. magnetický moment čí spin). Tudíž je suma aditivních kvantových čísel systému částice-antičástice nulová. V případě, že se hmota srazí s odpovídající antihmotou, vzájemně se vyruší -- annihilují. Při tomto procesu se uvolní značné množství energie, umožňující vznik nových párů částice-antičástice.

Po objevu pozitronu se začali vědci zabývat myšlenkou, zda i další elektricky nabité částice mají svého "antikamaráda". Až v roce 1955 byla konečně potvrzena existence antiprotonu \cite{chamberlain_1955}. Další metou se logicky stal objev antineutronu. Pozitron od elektronu, nebo antiproton od protonu již na první pohled odlišuje opačný náboj. Nicméně náboj neutronu je přeci nulový, může tedy antineutron vůbec existovat? Háček je v tom, že neutron má sice nulový náboj, ale zároveň nenulový magnetický moment. Antineutron tedy lze poznat podle magnetického momentu, který je opačný, nežli u neutronu. K jeho prvímu pozorování došlo rok po objevu antiprotonu \cite{cork}.

Jelikož elektron, proton a neutron se běžně vyskytují ve vázaných stavech, nabídla se domněnka, zda by i jejich antičástice nemohly obdobně koexistovat. Skutečně, v roce 1995 se podařilo na urychlovači LEAR\footnote{předchůdce dnešního antiprotonového zpomalovače (AD)} v CERNu poprvé připravit atom antivodíku \cite{lear}. Kvůli jeho krátké době života však nebylo tehdy možné jej dále použít v experimentu.
Za zmínku také stojí možnost vytvoření systémů kombinujících částice s antičísticemi. %, v niž byla hmota vázána s antihmotou, např. 
Pozorováno bylo např. pozitronium (vázaný stav elektronu a pozitronu) nebo heliový atom, ve kterém byl jeden eletron nahrazen antiprotonem.



% end section
%%%%%%%%%%%%%%%%%%%%%%%%%%%%%%%%%%%%%%%%%%%%%%%%%%%%%%%%%%%%%%%%%%%%%%%%%%%%%%%%%%%%%%%%%%%%%%%%%%%%%%%%%%

\section*{Proč je ve Vesmíru více hmoty nežli antihmoty?} 

	\begin{itemize}
	\item B mezony,
	\item AMS: pokud potka nejake antijadra $\Rightarrow$ Dark Matter, anti-galaxie
	\end{itemize}

\todo[inline , backgroundcolor=green!60]{Hic sunt leones.}

% end section
%%%%%%%%%%%%%%%%%%%%%%%%%%%%%%%%%%%%%%%%%%%%%%%%%%%%%%%%%%%%%%%%%%%%%%%%%%%%%%%%%%%%%%%%%%%%%%%%%%%%%%%%%%

\section*{Hyperjádra}

\begin{equation}
 a = b+c
\end{equation}

$$
	\delta x = g T^2 .
$$
Pokud tedy změříme vertikální posunutí svazku procházejícího zařízením, můžeme ze zmíněného vzorce dopočítat velikost gravitačního zrychlení, které svazek pociťuje.


\end{document}
